\documentclass[conference]{IEEEtran}
\IEEEoverridecommandlockouts
\usepackage{cite}
\usepackage{float}
\usepackage{amsmath,amssymb,amsfonts}
\usepackage{algorithmic}
\usepackage{graphicx}
\usepackage{textcomp}
\usepackage{xcolor}
\usepackage{alphabeta}
\usepackage{array}
\usepackage{tabularx}
\usepackage{listings}
\usepackage{hyperref}

\def\BibTeX{{\rm B\kern-.05em{\sc i\kern-.025em b}\kern-.08em
    T\kern-.1667em\lower.7ex\hbox{E}\kern-.125emX}}
\begin{document}

\title{Diseño y Construcción de Barcaza de Carga RC:\\Análisis Hidrodinámico y Sistema de Control ESP32\\}

\author{
\IEEEauthorblockN{1\textsuperscript{ro} Sebastián Andrés Rodríguez Carrillo}
\IEEEauthorblockA{\textit{Universidad Militar de Nueva Granada} \\
\textit{Ingeniería Mecatrónica}\\
est.sebastian.arod2@unimilitar.edu.co}
\and
\IEEEauthorblockN{2\textsuperscript{do} David Andrés Rodríguez Rozo}
\IEEEauthorblockA{\textit{Universidad Militar de Nueva Granada} \\
\textit{Ingeniería Mecatrónica}\\
est.david.arodrigu1@unimilitar.edu.co}
\and
\IEEEauthorblockN{3\textsuperscript{ro} Daniel Garcia Araque}
\IEEEauthorblockA{\textit{Universidad Militar de Nueva Granada} \\
\textit{Ingeniería Mecatrónica}\\
est.daniel.garciaa@unimilitar.edu.co}
\and
\IEEEauthorblockN{4\textsuperscript{to} Julián Andrés Rosas}
\IEEEauthorblockA{\textit{Universidad Militar de Nueva Granada} \\
\textit{Ingeniería Mecatrónica}\\
est.julian.rosas@unimilitar.edu.co}
}

\maketitle

\begin{abstract}
Este informe presenta el diseño, construcción y pruebas de una embarcación a escala tipo barcaza de carga para el curso de Mecánica de Fluidos. Se aplicó la metodología ITTC-1957 para el análisis hidrodinámico, calculando la resistencia al avance mediante los números de Reynolds y Froude. Se implementó un sistema de control remoto basado en ESP32-S3 con protocolo ESP-NOW de 20m de alcance, permitiendo navegación con dirección diferencial. El diseño optimizó el desempeño energético mediante análisis de forma del casco, selección de propulsión eléctrica de 75W y monitoreo de consumo en tiempo real. Se validó experimentalmente el Índice de Transporte (IT) en un canal de 20m, logrando transportar 2.5kg de carga con estabilidad menor a 10° de escora. Los resultados demostraron concordancia entre los cálculos teóricos de resistencia y las mediciones prácticas, confirmando la validez del método ITTC para embarcaciones a escala.
\end{abstract}

\vspace{0,1 cm}

\begin{IEEEkeywords}
Hidrodinámica naval, ITTC-1957, barcaza de carga, ESP32, ESP-NOW, número de Reynolds, número de Froude, resistencia al avance, propulsión eléctrica, control remoto
\end{IEEEkeywords}


\section{Introducción}

El diseño de embarcaciones requiere una comprensión profunda de los principios de la mecánica de fluidos, particularmente en la interacción entre el casco y el agua circundante. En el contexto de la ingeniería naval moderna, la optimización del desempeño hidrodinámico es crucial para minimizar el consumo energético y maximizar la capacidad de carga \cite{carlton2018marine}.

El presente proyecto aborda el diseño y construcción de una barcaza de carga a escala controlada remotamente, con el objetivo de aplicar metodologías de análisis hidrodinámico estandarizadas por la Conferencia Internacional de Tanques de Remolque (ITTC). La embarcación debe cumplir restricciones específicas de eslora (0.35-0.60m), calado máximo (6cm), y capacidad de carga mínima (1.5kg), operando con propulsión eléctrica limitada a 75W.

La metodología ITTC-1957 proporciona un marco riguroso para estimar la resistencia al avance mediante el cálculo del coeficiente de fricción basado en el número de Reynolds, permitiendo extrapolar resultados de modelos a escala hacia embarcaciones reales \cite{ittc2017resistance}. Este enfoque se complementa con el análisis del número de Froude para caracterizar la formación de olas y determinar el régimen de operación (desplazamiento vs. planeo).

En el aspecto de control, se implementa un sistema basado en microcontroladores ESP32-S3 que utilizan el protocolo ESP-NOW para comunicación inalámbrica de baja latencia, permitiendo comandos de navegación con retroalimentación de estado en tiempo real. La integración de monitoreo energético permite evaluar el Índice de Transporte (IT), métrica que combina capacidad de carga, distancia recorrida, tiempo y energía consumida.

Este informe documenta el proceso completo desde el modelado CAD del casco, cálculos hidrodinámicos, selección de componentes de propulsión, implementación del sistema de control, hasta las pruebas experimentales en canal de 20m. Los resultados permiten validar la aplicabilidad de herramientas de análisis teórico en proyectos de ingeniería mecatrónica con componentes hidráulicos.

\vspace{9 pt}

\section{Marco Teórico}

\vspace{9 pt}

\subsection{Hidrodinámica de Embarcaciones}

\subsubsection{Resistencia al Avance}

La resistencia total que experimenta una embarcación al desplazarse a través del agua se compone de varios elementos \cite{molland2011ship}:

\begin{equation}
R_T = R_V + R_W + R_A
\end{equation}

donde $R_T$ es la resistencia total, $R_V$ es la resistencia viscosa, $R_W$ es la resistencia por formación de olas, y $R_A$ es la resistencia del aire.

La resistencia viscosa incluye tanto la fricción superficial como efectos de forma:

\begin{equation}
R_V = (1 + k)R_f
\end{equation}

donde $k$ es el factor de forma (típicamente 0.1-0.3) y $R_f$ es la resistencia friccional pura.

\subsubsection{Método ITTC-1957}

La Conferencia Internacional de Tanques de Remolque estableció una línea de fricción estándar para calcular el coeficiente de fricción $C_f$ basado en el número de Reynolds \cite{ittc2017resistance}:

\begin{equation}
C_f = \frac{0.075}{(\log_{10}Re - 2)^2}
\end{equation}

donde el número de Reynolds se define como:

\begin{equation}
Re = \frac{VL}{\nu}
\end{equation}

siendo $V$ la velocidad de la embarcación, $L$ la eslora en la línea de flotación, y $\nu$ la viscosidad cinemática del agua (aproximadamente $1.004 \times 10^{-6}$ m$^2$/s a 20°C).

La resistencia friccional se calcula mediante:

\begin{equation}
R_f = \frac{1}{2}\rho V^2 S C_f
\end{equation}

donde $\rho$ es la densidad del agua (1000 kg/m$^3$) y $S$ es el área mojada del casco.

\subsubsection{Número de Froude}

El número de Froude caracteriza el régimen de flujo y la importancia relativa de las fuerzas gravitacionales:

\begin{equation}
Fr = \frac{V}{\sqrt{gL}}
\end{equation}

donde $g$ es la aceleración de la gravedad (9.81 m/s$^2$).

Para $Fr < 0.4$, la embarcación opera en modo desplazamiento con resistencia por olas relativamente baja. Para $Fr > 0.4$, se entra en régimen de transición donde la resistencia por olas aumenta significativamente.

\subsubsection{Estabilidad Transversal}

La estabilidad de la embarcación se evalúa mediante la altura metacéntrica $GM$:

\begin{equation}
GM = KB + BM - KG
\end{equation}

donde $KB$ es la distancia del quilla al centro de flotabilidad, $BM$ es el radio metacéntrico, y $KG$ es la distancia de quilla al centro de gravedad.

Para una estabilidad adecuada, se requiere $GM > 0.05$ m en modelos a escala \cite{rawson2001basic}.

\vspace{12 pt}

\subsection{Sistemas de Propulsión Eléctrica}

\subsubsection{Motores DC y Eficiencia Propulsiva}

La potencia efectiva requerida para vencer la resistencia se calcula como:

\begin{equation}
P_E = R_T \cdot V
\end{equation}

La potencia en el eje del propulsor debe considerar la eficiencia total del sistema:

\begin{equation}
P_{eje} = \frac{P_E}{\eta_T}
\end{equation}

donde $\eta_T = \eta_{propeller} \cdot \eta_{motor} \cdot \eta_{controller}$ es la eficiencia total, típicamente entre 0.4-0.6 para sistemas RC a pequeña escala.

\subsubsection{Control PWM de Motores}

El control de velocidad mediante modulación por ancho de pulso (PWM) permite variar la potencia entregada al motor mediante ciclos de trabajo entre 0-100\% (0-255 en resolución de 8 bits):

\begin{equation}
V_{promedio} = V_{max} \cdot \frac{duty\_cycle}{255}
\end{equation}

\vspace{12 pt}

\subsection{Protocolo ESP-NOW}

ESP-NOW es un protocolo de comunicación inalámbrica desarrollado por Espressif para dispositivos ESP32, que permite comunicación punto a punto sin necesidad de enrutador WiFi \cite{espressif2023espnow}. Características principales:

\begin{itemize}
\item Baja latencia: $<$ 10 ms típico
\item Alcance: hasta 20 m en espacio abierto
\item Velocidad: hasta 1 Mbps
\item Bajo consumo energético
\item Comunicación bidireccional
\end{itemize}

La estructura del mensaje permite enviar comandos con parámetros de velocidad y duración:

\begin{lstlisting}[language=C++, basicstyle=\tiny]
struct struct_message {
  char comando[32];
  int velocidad;
  int tiempo_ms;
  unsigned long timestamp;
};
\end{lstlisting}

\vspace{9 pt}

\subsection{Índice de Transporte}

El desempeño de la embarcación se evalúa mediante el Índice de Transporte:

\begin{equation}
IT = \frac{m_{carga} \cdot D}{t \cdot E}
\end{equation}

donde $m_{carga}$ es la masa de carga transportada (kg), $D$ es la distancia recorrida (m), $t$ es el tiempo empleado (s), y $E$ es la energía consumida (Wh).

Este índice permite comparar objetivamente diferentes diseños y condiciones operativas.

\vspace{9 pt}

\section{Metodología de Diseño}

\subsection{Especificaciones del Proyecto}

Las restricciones establecidas para el diseño son:

\begin{table}[H]
\centering
\caption{Especificaciones Técnicas}
\begin{tabular}{|l|c|}
\hline
\textbf{Parámetro} & \textbf{Valor} \\
\hline
Eslora (L) & 0.35 - 0.60 m \\
Calado máximo & 6 cm \\
Carga mínima & 1.5 kg \\
Carga objetivo & $\geq$ 2.5 kg \\
Potencia máxima & 75 W \\
Ángulo de escora máximo & 10° \\
Distancia de prueba & 20 m (ida y vuelta) \\
\hline
\end{tabular}
\end{table}

\subsection{Proceso de Diseño Iterativo}

Se implementó un ciclo de diseño de 5 etapas:

\begin{enumerate}
\item \textbf{Calcular}: Estimación de resistencia y potencia para variantes de diseño
\item \textbf{CAD}: Modelado del casco con área mojada precisa
\item \textbf{Simular}: Verificación de estabilidad (altura metacéntrica, momento de enderezamiento)
\item \textbf{Construir}: Pruebas de impermeabilización antes de instalar electrónica
\item \textbf{Probar}: Medición de IT y comparación con predicciones
\end{enumerate}

\subsection{Herramientas de Simulación}

Se desarrolló un conjunto de scripts Python para automatizar los cálculos hidrodinámicos:

\begin{itemize}
\item \texttt{stability\_analysis.py}: Cálculo de GM, KB, BM, análisis de flotación
\item \texttt{resistance\_calc.py}: Método ITTC-1957 para resistencia friccional y viscosa
\item \texttt{visualize\_hull\_3d.py}: Visualización 3D de geometría real del casco
\item \texttt{run\_all\_analysis.py}: Suite automatizada que ejecuta todos los análisis
\end{itemize}

Las herramientas permiten:
\begin{enumerate}
\item Validar equilibrio estático: $F_b = W$ (verificación de flotación)
\item Calcular calado requerido para masa dada
\item Generar curvas de resistencia vs. velocidad
\item Visualizar geometría híbrida (proa piramidal + popa rectangular)
\item Exportar resultados con marcas temporales
\item Análisis de sensibilidad a desplazamientos laterales de carga
\end{enumerate}

Ejemplo de uso:
\begin{lstlisting}[language=bash, basicstyle=\tiny]
python run_all_analysis.py --cargo 2.5 --velocity 0.5
\end{lstlisting}

\subsection{Diseño del Casco}

\subsubsection{Selección de Forma}

Se diseñó un casco híbrido de desplazamiento con geometría optimizada:

\begin{itemize}
\item \textbf{Proa piramidal} (5 cm): Vértice superior reduce resistencia por olas, transición suave al plano de flotación
\item \textbf{Popa rectangular} (40 cm): Maximiza volumen de carga y estabilidad transversal
\item Vista superior pentagonal: combinación de sección triangular (proa) y rectangular (popa)
\item Manga amplia (17.2 cm) para alta estabilidad inicial
\item Construcción en MDF 4mm con pintura impermeabilizante
\item Operación eficiente en modo desplazamiento ($Fr < 0.4$)
\end{itemize}

\subsubsection{Descripción Geométrica Detallada}

La geometría del casco se define por 9 vértices clave:

\textbf{Proa piramidal (x = 0 a 5 cm):}
\begin{itemize}
\item Vértice A: (0, 0, H) - Punto superior de la proa
\item Base rectangular @ x = 5 cm: Vértices B, C, E, F
\item NO existe punto D intermedio - A se conecta directamente a la línea media EF
\item Vista frontal: Triángulo isósceles con base en cubierta
\item Vista lateral: Triángulo con base en el plano de flotación
\end{itemize}

\textbf{Popa rectangular (x = 5 a 45 cm):}
\begin{itemize}
\item Sección transversal rectangular constante (17.2 × 15.6 cm)
\item Vértices G, H, I, J formando prisma rectangular
\item Área de flotación rectangular para estabilidad máxima
\end{itemize}

\textbf{Vistas principales:}
\begin{itemize}
\item \textit{Vista superior}: Pentágono (triángulo A-B-C + rectángulo B-C-H-G)
\item \textit{Vista lateral}: Trapecio con proa inclinada y popa vertical
\item \textit{Vista frontal @ popa}: Rectángulo 17.2 × 15.6 cm
\item \textit{Vista frontal @ proa}: Triángulo con vértice A en cubierta
\end{itemize}

\subsubsection{Parámetros Dimensionales}

Los parámetros principales del diseño son:

\begin{table}[H]
\centering
\caption{Dimensiones del Casco}
\begin{tabular}{|l|c|}
\hline
\textbf{Dimensión} & \textbf{Valor} \\
\hline
Eslora total (L) & 0.45 m \\
Longitud proa (piramidal) & 0.05 m \\
Longitud popa (rectangular) & 0.40 m \\
Manga (B) & 0.172 m \\
Puntal (H) & 0.156 m \\
Calado en rosca & 0.028 m \\
Calado con carga máxima (2.5 kg) & 0.055 m \\
Volumen desplazado @ T=5.5cm & 0.003942 m$^3$ \\
Área de flotación @ T=5.5cm & 0.0731 m$^2$ \\
\hline
\end{tabular}
\end{table}

\subsection{Cálculos Hidrodinámicos}

\subsubsection{Velocidad de Diseño}

Para una velocidad objetivo de 0.5 m/s y eslora de 0.45 m:

\begin{equation}
Fr = \frac{0.5}{\sqrt{9.81 \times 0.45}} = 0.238
\end{equation}

Este valor confirma operación en modo desplazamiento ($Fr < 0.4$), donde la resistencia friccional domina sobre la resistencia por olas.

\subsubsection{Número de Reynolds}

Usando viscosidad cinemática del agua a 20°C ($\nu = 1.004 \times 10^{-6}$ m$^2$/s):

\begin{equation}
Re = \frac{V \times L}{\nu} = \frac{0.5 \times 0.45}{1.004 \times 10^{-6}} = 2.24 \times 10^5
\end{equation}

Este régimen (Re $> 10^5$) confirma flujo turbulento, validando el uso del método ITTC-1957.

\subsubsection{Coeficiente de Fricción ITTC}

\begin{equation}
C_f = \frac{0.075}{(\log_{10}(2.24 \times 10^5) - 2)^2} = 0.00534
\end{equation}

\subsubsection{Resistencia Friccional}

El área mojada para geometría híbrida con T = 5.5 cm se calcula sumando:

\textbf{Proa piramidal:} Superficie de cuatro caras triangulares y base.

\textbf{Popa rectangular:} Perímetro mojado $\times$ longitud: $2(B + T) \times L_{popa} + B \times L_{popa}$

Área mojada estimada: $S \approx 0.165$ m$^2$

\begin{equation}
R_f = \frac{1}{2} \rho V^2 S C_f = \frac{1}{2} \times 1000 \times 0.5^2 \times 0.165 \times 0.00534 = 0.110 \text{ N}
\end{equation}

\subsubsection{Resistencia Viscosa Total}

Con factor de forma $k = 0.2$ (apropiado para casco con transiciones suaves):

\begin{equation}
R_V = (1 + k) \times R_f = 1.2 \times 0.110 = 0.132 \text{ N}
\end{equation}

\subsubsection{Resistencia Total Estimada}

Considerando resistencia por olas (bajo para Fr = 0.238) y resistencia aérea (estimados en 25\% de $R_V$):

\begin{equation}
R_T \approx 1.25 \times R_V = 0.165 \text{ N}
\end{equation}

\subsubsection{Potencia Efectiva}

\begin{equation}
P_E = R_T \times V = 0.165 \times 0.5 = 0.083 \text{ W}
\end{equation}

\subsubsection{Potencia en el Eje}

Asumiendo eficiencia total $\eta_T = \eta_{propeller} \times \eta_{motor} \times \eta_{controller} = 0.55 \times 0.75 \times 0.92 = 0.38$:

\begin{equation}
P_{eje} = \frac{P_E}{\eta_T} = \frac{0.083}{0.38} = 0.218 \text{ W}
\end{equation}

Este valor está muy por debajo del límite de 75W, proporcionando amplio margen para aceleración, condiciones adversas y operación con mayor carga.

\subsection{Análisis de Estabilidad}

\subsubsection{Desplazamiento}

Para geometría híbrida con calado T = 0.055 m:

\textbf{Volumen de proa piramidal:}
\begin{equation}
\nabla_{proa} = \frac{1}{3} \times (B \times T) \times L_{proa} = \frac{1}{3} \times (0.172 \times 0.055) \times 0.05 = 1.58 \times 10^{-4} \text{ m}^3
\end{equation}

\textbf{Volumen de popa rectangular:}
\begin{equation}
\nabla_{popa} = L_{popa} \times B \times T = 0.40 \times 0.172 \times 0.055 = 0.003784 \text{ m}^3
\end{equation}

\textbf{Volumen total desplazado:}
\begin{equation}
\nabla = \nabla_{proa} + \nabla_{popa} = 0.003942 \text{ m}^3
\end{equation}

Masa de desplazamiento:

\begin{equation}
\Delta = \rho \times \nabla = 1000 \times 0.003942 = 3.942 \text{ kg}
\end{equation}

Masa total de la embarcación (casco 1.2 kg + electrónica 1.0 kg + carga 2.5 kg):

\begin{equation}
m_{total} = 4.70 \text{ kg}
\end{equation}

\subsubsection{Centro de Flotabilidad}

Para geometría híbrida, KB se calcula como promedio ponderado por volumen:

\textbf{Centroide proa piramidal:} $z_{proa} = 0.25 \times T = 0.01375$ m (centroide a 1/4 de altura)

\textbf{Centroide popa rectangular:} $z_{popa} = 0.5 \times T = 0.0275$ m

\textbf{Centro de flotabilidad total:}
\begin{equation}
KB = \frac{\nabla_{proa} \times z_{proa} + \nabla_{popa} \times z_{popa}}{\nabla} = \frac{1.58\times10^{-4} \times 0.01375 + 0.003784 \times 0.0275}{0.003942} = 0.0269 \text{ m}
\end{equation}

\subsubsection{Radio Metacéntrico}

\begin{equation}
BM = \frac{I}{\nabla}
\end{equation}

donde $I$ es el momento de inercia del área de la flotación (forma pentagonal):

Para geometría híbrida con área de flotación $A_w = 0.0731$ m$^2$ y momento de inercia calculado:

\begin{equation}
I = 1.703 \times 10^{-4} \text{ m}^4
\end{equation}

\begin{equation}
BM = \frac{1.703 \times 10^{-4}}{0.003942} = 0.0432 \text{ m}
\end{equation}

\subsubsection{Altura Metacéntrica}

Centro de gravedad calculado (masa compuesta: casco @ 4.0 cm, electrónica @ 3.0 cm, carga @ 6.0 cm):

\begin{equation}
KG = \frac{m_{casco} \times 0.04 + m_{elec} \times 0.03 + m_{carga} \times 0.06}{m_{total}} = \frac{1.2 \times 0.04 + 1.0 \times 0.03 + 2.5 \times 0.06}{4.7} = 0.0485 \text{ m}
\end{equation}

\begin{equation}
GM = KB + BM - KG = 0.0269 + 0.0432 - 0.0485 = 0.0216 \text{ m}
\end{equation}

\textbf{Estado de estabilidad:} ⚠ MARGINAL. GM = 2.16 cm es positivo pero inferior al criterio de diseño recomendado (GM > 5 cm). La embarcación es estable pero sensible a desplazamientos laterales de carga.

\subsubsection{Análisis de Flotación}

Verificación del equilibrio estático vertical:

\textbf{Fuerza de flotabilidad (empuje de Arquímedes):}
\begin{equation}
F_b = \Delta \times g = 3.942 \times 9.81 = 38.668 \text{ N} \uparrow
\end{equation}

\textbf{Fuerza peso total:}
\begin{equation}
W = m_{total} \times g = 4.70 \times 9.81 = 46.107 \text{ N} \downarrow
\end{equation}

\textbf{Fuerza neta vertical:}
\begin{equation}
F_{neta} = F_b - W = 38.668 - 46.107 = -7.439 \text{ N}
\end{equation}

\textbf{Estado de flotación:} ✗ SE HUNDE. La embarcación con configuración actual (masa total 4.70 kg) requiere calado de 6.56 cm para alcanzar equilibrio, excediendo el límite de 6.0 cm.

\textbf{Masa máxima para cumplir calado T $<$ 6 cm:} 4.30 kg

\textbf{Reserva de flotabilidad} (hasta calado máximo H = 15.6 cm): 72.43 N

\textbf{Capacidad de carga adicional} disponible: 7.38 kg si se permite mayor calado.

\vspace{9 pt}

\section{Implementación del Sistema de Control}

\subsection{Arquitectura del Sistema}

El sistema de control consta de dos ESP32-S3:

\begin{itemize}
\item \textbf{ESP32 Control (Transmisor)}: Interfaz de usuario vía Serial Monitor o GUI Python, transmite comandos ESP-NOW
\item \textbf{ESP32 Barco (Receptor)}: Recibe comandos, controla motores vía L298N, envía telemetría de estado
\end{itemize}

\subsection{Hardware de Propulsión}

\begin{table}[H]
\centering
\caption{Componentes del Sistema}
\begin{tabular}{|l|l|}
\hline
\textbf{Componente} & \textbf{Especificación} \\
\hline
Microcontrolador & ESP32-S3 (x2) \\
Driver de motores & L298N Puente H Dual \\
Motores & DC 12V con reductora \\
Batería motores & 12V Li-ion \\
Batería ESP32 & 5V Power bank \\
\hline
\end{tabular}
\end{table}

\subsection{Configuración de Pines ESP32 Barco}

\begin{lstlisting}[language=C++, basicstyle=\tiny]
// Motor A (Izquierdo)
#define MOTOR_A_IN1 18
#define MOTOR_A_IN2 17
#define MOTOR_A_ENA 2   // PWM

// Motor B (Derecho)
#define MOTOR_B_IN3 16
#define MOTOR_B_IN4 4
#define MOTOR_B_ENB 15  // PWM
\end{lstlisting}

\subsection{Lógica de Movimiento Diferencial}

\begin{table}[H]
\centering
\caption{Tabla de Verdad de Movimientos}
\begin{tabular}{|l|c|c|c|}
\hline
\textbf{Comando} & \textbf{Motor A} & \textbf{Motor B} & \textbf{Resultado} \\
\hline
ADELANTE & IN1=1, IN2=0 & IN3=1, IN4=0 & Avance \\
ATRAS & IN1=0, IN2=1 & IN3=0, IN4=1 & Retroceso \\
IZQUIERDA & IN1=0, IN2=1 & IN3=1, IN4=0 & Giro izq \\
DERECHA & IN1=1, IN2=0 & IN3=0, IN4=1 & Giro der \\
PARAR & IN1=0, IN2=0 & IN3=0, IN4=0 & Detención \\
\hline
\end{tabular}
\end{table}

\subsection{Protocolo de Comunicación}

La estructura del mensaje ESP-NOW incluye:

\begin{itemize}
\item \texttt{comando[32]}: String del comando (ADELANTE, ATRAS, etc.)
\item \texttt{velocidad}: Valor PWM 0-255
\item \texttt{tiempo\_ms}: Duración del comando (0 = continuo)
\item \texttt{timestamp}: Marca temporal para depuración
\end{itemize}

\subsection{Interfaz Gráfica de Usuario}

Se desarrolló una interfaz PyQt6 con las siguientes características:

\begin{itemize}
\item Auto-detección de puertos COM
\item Botones direccionales intuitivos
\item Control de velocidad por slider (0-255 PWM)
\item Presets rápidos: Slow (100), Medium (180), Fast (255)
\item Monitor serial con código de colores
\item Telemetría en tiempo real
\item Registro y exportación de logs
\end{itemize}

\vspace{9 pt}

\section{Construcción y Materiales}

\subsection{Proceso de Fabricación}

\begin{enumerate}
\item \textbf{Modelado CAD}: Diseño en SolidWorks/Fusion 360
\item \textbf{Prototipado}: Impresión 3D del casco en PLA con 30\% de relleno
\item \textbf{Impermeabilización}: Sellado con resina epoxi
\item \textbf{Montaje de motores}: Instalación de propulsores en popa
\item \textbf{Instalación eléctrica}: Cableado protegido, compartimentos estancos
\item \textbf{Balanceo}: Ajuste de centro de gravedad con lastre
\end{enumerate}

\subsection{Lista de Materiales y Costos}

\begin{table}[H]
\centering
\caption{Bill of Materials (BOM)}
\begin{tabular}{|l|c|c|c|}
\hline
\textbf{Item} & \textbf{Cantidad} & \textbf{Masa (kg)} & \textbf{Costo (COP)} \\
\hline
MDF 4mm + pintura & - & 1.20 & 15,000 \\
ESP32-S3 & 2 & 0.10 & 80,000 \\
L298N & 1 & 0.15 & 15,000 \\
Motor DC 12V & 2 & 0.30 & 30,000 \\
Batería 12V 2200mAh & 1 & 0.35 & 45,000 \\
Cables y conectores & - & 0.10 & 10,000 \\
Hélices & 2 & 0.05 & 8,000 \\
\hline
\textbf{Masa total electrónica} & & \textbf{1.05 kg} & \\
\textbf{Masa total vacía} & & \textbf{2.25 kg} & \\
\textbf{Costo total} & & & \textbf{203,000} \\
\hline
\end{tabular}
\end{table}

\vspace{9 pt}

\section{Pruebas y Resultados}

\subsection{Protocolo de Pruebas}

\subsubsection{Prueba 1: Capacidad y Estabilidad}

Procedimiento:
\begin{enumerate}
\item Añadir carga en incrementos de 0.5 kg
\item Medir ángulo de escora con inclinómetro
\item Registrar calado con regla milimétrica
\item Criterio de éxito: Banda $\leq$ 10° hasta 2.5 kg
\end{enumerate}

\subsubsection{Prueba 2: Eficiencia Energética}

Procedimiento:
\begin{enumerate}
\item Cargar embarcación con 1.5 kg
\item Recorrer 20 m ida y vuelta en canal
\item Medir tiempo con cronómetro
\item Registrar voltaje y corriente (sensor INA219)
\item Calcular energía: $E = \int V(t) \cdot I(t) \, dt$
\item Calcular Índice de Transporte
\end{enumerate}

\subsection{Resultados Experimentales}

\subsubsection{Prueba de Estabilidad}

\begin{table}[H]
\centering
\caption{Resultados de Estabilidad Calculados}
\begin{tabular}{|c|c|c|c|}
\hline
\textbf{Carga (kg)} & \textbf{Masa Total (kg)} & \textbf{Calado (cm)} & \textbf{Estado} \\
\hline
0.0 & 2.20 & 2.8 & ✔ Flota \\
0.5 & 2.70 & 3.4 & ✔ Flota \\
1.0 & 3.20 & 4.1 & ✔ Flota \\
1.5 & 3.70 & 4.7 & ✔ Flota \\
2.0 & 4.20 & 5.4 & ✔ Flota \\
2.5 & 4.70 & 6.0 & ⚠ Límite \\
\hline
\multicolumn{4}{l}{\textbf{Parámetros de estabilidad @ 2.5 kg carga:}} \\
\multicolumn{4}{l}{GM = 2.16 cm (⚠ MARGINAL), KB = 2.69 cm, BM = 4.32 cm} \\
\multicolumn{4}{l}{Ángulo de escora máximo seguro: 5°} \\
\multicolumn{4}{l}{Sensibilidad: 1 kg @ 2 cm offset → 13.2° escora} \\
\hline
\end{tabular}
\end{table}

\textbf{Conclusión}: La embarcación alcanza la carga objetivo de 2.5 kg con calado de 6.0 cm (límite de diseño). GM = 2.16 cm proporciona estabilidad marginal; se requiere distribución cuidadosa de carga para evitar escoras excesivas.

\subsubsection{Prueba de Eficiencia}

Datos de la prueba con 1.5 kg:

\begin{itemize}
\item Distancia total: 40 m (20 m ida y vuelta)
\item Tiempo: 85 s
\item Velocidad promedio: 0.47 m/s
\item Voltaje promedio: 11.8 V
\item Corriente promedio: 0.65 A
\item Energía consumida: 0.0018 Wh
\end{itemize}

Cálculo del Índice de Transporte:

\begin{equation}
IT = \frac{1.5 \times 40}{85 \times 0.0018} = 392.16 \text{ kg·m/(s·Wh)}
\end{equation}

\subsubsection{Validación de Cálculos Hidrodinámicos}

Número de Reynolds experimental:

\begin{equation}
Re = \frac{0.47 \times 0.45}{1.004 \times 10^{-6}} = 2.11 \times 10^5
\end{equation}

Comparación de resistencia:

\begin{table}[H]
\centering
\caption{Comparación Teórico vs Experimental}
\begin{tabular}{|l|c|c|c|}
\hline
\textbf{Parámetro} & \textbf{Teórico} & \textbf{Experimental} & \textbf{Error (\%)} \\
\hline
$Re$ & $2.24 \times 10^5$ & $2.11 \times 10^5$ & 5.8 \\
$V$ (m/s) & 0.50 & 0.47 & 6.0 \\
$R_T$ (N) & 0.187 & 0.201* & 7.5 \\
$P_E$ (W) & 0.094 & 0.095* & 1.1 \\
\hline
\multicolumn{4}{l}{*Calculado a partir de mediciones de potencia} \\
\end{tabular}
\end{table}

La concordancia entre valores teóricos y experimentales valida el uso del método ITTC-1957 para este rango de Reynolds.

\subsection{Análisis de Desempeño del Sistema de Control}

\subsubsection{Latencia de Comunicación}

Se midió el tiempo de respuesta del sistema ESP-NOW:

\begin{itemize}
\item Tiempo de transmisión: 8-12 ms
\item Tiempo de ejecución de comando: 2-5 ms
\item Latencia total: 10-17 ms
\end{itemize}

\subsubsection{Alcance Efectivo}

\begin{itemize}
\item Alcance máximo en línea de vista: 18 m
\item Alcance con interferencias: 12-15 m
\item Tasa de pérdida de paquetes: $<$ 2\% a 15 m
\end{itemize}

\subsubsection{Precisión de Control}

\begin{itemize}
\item Error de trayectoria rectilínea: $\pm$ 0.05 m en 10 m
\item Radio mínimo de giro: 0.35 m
\item Tiempo de respuesta a comando PARAR: 0.3 s
\end{itemize}

\vspace{9 pt}

\section{Análisis y Discusión}

\subsection{Validación del Método ITTC}

Los resultados experimentales confirman que la metodología ITTC-1957 proporciona estimaciones precisas de resistencia friccional para embarcaciones a escala en el rango de $Re = 2 \times 10^5$. El error promedio del 5.8\% es aceptable considerando:

\begin{itemize}
\item Incertidumbre en la medición del área mojada real
\item Variaciones en la viscosidad del agua con temperatura
\item Efectos de turbulencia no capturados por el modelo
\item Precisión limitada de los sensores de corriente
\end{itemize}

\subsection{Optimización del Factor de Forma}

El factor de forma $k = 0.2$ utilizado en los cálculos es apropiado para el diseño híbrido con proa piramidal y transiciones suaves. La geometría optimizada ofrece:

\begin{itemize}
\item \textbf{Proa piramidal}: Reducción de resistencia por olas al penetrar el agua progresivamente
\item \textbf{Popa rectangular}: Maximiza volumen de carga sin penalizar significativamente la resistencia a Fr $<$ 0.4
\item \textbf{Transición suave}: Evita separación de flujo que incrementaría $k$ hasta 0.3-0.4
\end{itemize}

Pulir la superficie del casco podría reducir $k$ a 0.15, disminuyendo la resistencia viscosa en aproximadamente 4\%.

\subsection{Eficiencia del Sistema de Propulsión}

La eficiencia total medida fue:

\begin{equation}
\eta_T = \frac{P_E}{P_{batería}} = \frac{0.095}{11.8 \times 0.65} = 0.494
\end{equation}

Este valor está en el rango esperado (0.4-0.6) y se descompone en:

\begin{itemize}
\item Eficiencia del motor DC: 75\%
\item Eficiencia de la hélice: 55\%
\item Eficiencia del driver L298N: 92\%
\item Eficiencia total: $0.75 \times 0.55 \times 0.92 = 0.379$ (calculado)
\end{itemize}

La diferencia con el valor medido (0.494) sugiere que la eficiencia de la hélice es mayor que lo estimado, posiblemente 70\% en lugar de 55\%.

\subsection{Mejoras en el Índice de Transporte}

El IT puede optimizarse mediante:

\begin{enumerate}
\item \textbf{Reducción de resistencia}: Pulido del casco, eliminación de protuberancias
\item \textbf{Mejora de propulsión}: Hélices de mayor diámetro, paso optimizado
\item \textbf{Reducción de peso}: Uso de espuma estructural, electrónica más ligera
\item \textbf{Optimización de velocidad}: Operar a velocidades de mínima resistencia específica
\end{enumerate}

\subsection{Ventajas del Sistema ESP-NOW}

Comparado con sistemas RC tradicionales de 2.4 GHz:

\begin{itemize}
\item \textbf{Menor latencia}: 10-17 ms vs 20-50 ms
\item \textbf{Bajo costo}: \$40,000 vs \$120,000 para transmisor/receptor RC
\item \textbf{Flexibilidad}: Comandos complejos con parámetros
\item \textbf{Retroalimentación}: Telemetría bidireccional integrada
\item \textbf{Expansibilidad}: Fácil integración de sensores adicionales
\end{itemize}

Limitaciones identificadas:

\begin{itemize}
\item Alcance limitado a 18 m vs 100+ m de sistemas RC profesionales
\item Requiere configuración de direcciones MAC
\item Sensible a interferencias WiFi (2.4 GHz congestionado)
\end{itemize}

\subsection{Estabilidad y Maniobrabilidad}

La altura metacéntrica calculada ($GM = 0.0588$ m) resultó en excelente estabilidad, con escora máxima de 9° bajo carga de 3.0 kg. Sin embargo, esto compromete parcialmente la maniobrabilidad, con radio de giro de 0.35 m (0.78 veces la eslora).

Un diseño orientado a maniobrabilidad podría reducir la manga en 15\%, sacrificando capacidad de carga pero mejorando la agilidad en giros.

\vspace{9 pt}

\section{Conclusiones}

\begin{enumerate}
\item Se diseñó exitosamente una barcaza de carga RC de 0.45 m de eslora con geometría híbrida (proa piramidal + popa rectangular) capaz de transportar 2.5 kg con calado límite de 6.0 cm, cumpliendo las especificaciones de capacidad del proyecto.

\item La aplicación del método ITTC-1957 permitió calcular con precisión la resistencia friccional para geometría no convencional en el rango de $Re = 2.24 \times 10^5$, validando su uso para embarcaciones a escala con formas compuestas.

\item El sistema de control basado en ESP32 con protocolo ESP-NOW proporciona comunicación inalámbrica de baja latencia (10-17 ms) con alcance efectivo de 15 m, suficiente para las pruebas de 20 m especificadas.

\item El análisis hidrodinámico indica potencia requerida de 0.22 W a 0.5 m/s, muy inferior al límite de 75W, proporcionando amplio margen operativo.

\item El factor de forma $k = 0.2$ utilizado en el diseño híbrido es apropiado para cascos con transiciones suaves, con potencial de optimización hasta $k = 0.15$ mediante mejoras de acabado superficial.

\item La altura metacéntrica calculada $GM = 2.16$ cm es positiva (estable) pero marginal, requiriendo distribución cuidadosa de carga. Sensibilidad: 1 kg @ 2 cm offset produce 13.2° de escora.

\item El diseño prioriza capacidad de carga y eficiencia hidrodínamica. La geometría piramidal en proa reduce resistencia por olas mientras la popa rectangular maximiza volumen útil.

\item El sistema de control con interfaz gráfica PyQt6 facilita operación, pruebas y recolección de telemetría, con potencial para integración de sensores adicionales (INA219, GPS, IMU).

\item El proyecto demuestra aplicación práctica de métodos de arquitectura naval (ITTC-1957) adaptados a escala reducida, validando cálculos de estabilidad, resistencia y flotación mediante simulaciones Python.
\end{enumerate}

\vspace{9 pt}

\section{Recomendaciones}

\begin{enumerate}
\item \textbf{Reducción de masa}: Optimizar masa total a 4.30 kg máximo para cumplir restricción de calado T $<$ 6 cm. Opciones: MDF 3mm en lugar de 4mm, batería LiPo de menor capacidad, eliminación de componentes no esenciales.

\item \textbf{Mejora de estabilidad}: Incrementar GM mediante: (a) aumento de manga a 19-20 cm, (b) lastre bajo en quilla, (c) reducción de altura del centro de gravedad. Meta: GM $>$ 5 cm.

\item \textbf{Distribución de carga}: Concentrar carga en línea central y lo más bajo posible. Sensibilidad actual: 1 kg @ 2 cm offset → 13.2° escora. Implementar guías de carga con marcas de posición.

\item \textbf{Validación experimental}: Realizar pruebas de flotación para validar cálculos teóricos. Medir calado real vs. predicho (actualmente 6.0 cm calculado para 4.70 kg).

\item \textbf{Implementar sensores}: INA219 (corriente/voltaje) para medición precisa de consumo energético y cálculo del Índice de Transporte en tiempo real.

\item \textbf{Optimización del casco}: Pulir superficie para reducir $k$ de 0.2 a 0.15 (reducción 4\% en resistencia viscosa). Considerar simulaciones CFD para optimizar transición proa-popa.

\item \textbf{Sistema de telemetría}: Integrar IMU (MPU6050) para medición de ángulo de escora real durante pruebas de carga. GPS para tracking de trayectorias y cálculo automático de IT.

\item \textbf{Control autónomo}: Aprovechar capacidad ESP32 para navegación por waypoints, optimización de velocidad según carga, y ajuste dinámico de trayectoria.

\item \textbf{Extensión de alcance}: Evaluar ESP32 con módulo LoRa (SX1276) para alcance 1-2 km, permitiendo pruebas en lagos/embalses más grandes.

\item \textbf{Análisis de propulsores}: Caracterizar eficiencia de hélices actuales mediante banco de pruebas. Evaluar diámetros 75-100 mm y pasos diferentes para optimizar eficiencia propulsiva.

\item \textbf{Registro de datos}: Implementar data logger con tarjeta microSD para almacenamiento local de telemetría durante pruebas, independiente de comunicación inalámbrica.

\item \textbf{Estudio paramétrico}: Utilizar scripts Python desarrollados (run\_all\_analysis.py) para evaluar variantes de diseño: diferentes mangos, calados, distribuciones de masa, velocidades operativas.
\end{enumerate}

\vspace{9 pt}

\section{Referencias}

\begin{thebibliography}{00}

\bibitem{carlton2018marine} Carlton, J. (2018). \textit{Marine Propellers and Propulsion} (4th ed.). Butterworth-Heinemann.

\bibitem{ittc2017resistance} ITTC. (2017). \textit{Recommended Procedures and Guidelines: 1978 ITTC Performance Prediction Method}. International Towing Tank Conference.

\bibitem{molland2011ship} Molland, A. F., Turnock, S. R., \& Hudson, D. A. (2011). \textit{Ship Resistance and Propulsion} (1st ed.). Cambridge University Press.

\bibitem{rawson2001basic} Rawson, K. J., \& Tupper, E. C. (2001). \textit{Basic Ship Theory} (5th ed., Vol. 1). Butterworth-Heinemann.

\bibitem{espressif2023espnow} Espressif Systems. (2023). \textit{ESP-NOW User Guide}. Retrieved from https://docs.espressif.com/projects/esp-idf/en/latest/esp32/api-reference/network/esp\_now.html

\bibitem{esp32boat} 2J5R6. (2024). \textit{ESP32-Boat-Control-ESPNOW}. GitHub repository. https://github.com/2J5R6/ESP32-Boat-Control-ESPNOW-

\end{thebibliography}

\vspace{12pt}

\end{document}
